\section{Introduction}
Control applications in automotive software systems often contain time-critical cause-effect chains.  
A time-critical chain typically includes the reading of sensor data, their processing and finally the control of actuators.  
Satisfying specified end-to-end deadlines of cause-effect  chains serves to ensure correct system behavior and can also increase driving comfort.
From an implementation point of view, a cause-effect chain consists of communicating tasks that are distributed among various system components. 
\smallskip

Designing a software application in such a way that all end-to-end deadlines of cause-effect chains are satisfied, even in the worst case, is challenging. 
Since software applications are regularly extended, it is another central concern to guarantee end-to-end deadlines in case of software updates without having to carry out a completely new software design. 


The tool \Tool (TOol to compute RObustness margins)
\begin{itemize}
	\item processes the model of a given software application with time-critical effect chains,
	\item calculates the maximum end-to-end latencies of the given effect chains,
	\item makes statements about the robustness of applications during software updates with regard to compliance with end-to-end deadlines.
\end{itemize}
\bigskip

\noindent
The results can be used
\begin{itemize}
	\item to evaluate the robustness of a given software application regarding software updates, 
	\item to identify critical (less robust) segments of a cause-effect chain,
	\item to predict whether a given software update will change the timing of effect chains such that end-to-end deadlines are missed,
	\item to determine where and why a given software update changes the time behavior such that end-to-end deadlines are missed.
\end{itemize}
\bigskip

\noindent
The tool \Tool is based on a novel robustness analysis that was developed in 2018/19 in a collaboration between Daimler and iTUBS.
The analysis is summarized in a scientific article which forms the appendix of this documentation.
The submission of the article to ECRTS 2020 is planned.

\subsection{Structure of the Document}
\textcolor[rgb]{0,0,1}{todo}