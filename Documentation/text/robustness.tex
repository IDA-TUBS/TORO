\section{Robustness}
\label{sec:robustness}

\subsection{Effect of Disturbances on Cause-Effect Chains}
%
A challenge related to software, which includes cause-effect chains, is to design it in a robust manner, so that the required upper bounds on end-to-end latencies are respected even in the presence of (a) static software extensions such as updates, patches etc. or (b) dynamic software extensions such as dynamic services in adaptive AUTOSAR. 
\smallskip

Software extensions may introduce entirely new tasks or increase worst case execution times of existing tasks.
Thus the amount of workload to be serviced by the system grows and slows down the execution of cause-effect chains which existed prior to the extension.
\smallskip

The timing impact of functional extensions can be seen as a \textbf{disturbance} with respect to a cause-effect chain of interest $\mathcal{C}$.
A \textbf{robust software design} should be able to tolerate a limited disturbance: meaning that if the disturbance occurs, the cause-effect chain $\mathcal{C}$ can meet its end-to-end deadline even in the worst case. 
We specify the ability of the software to tolerate disturbances by a \textbf{robustness margin}, a concept which will be discussed later in detail.
Furthermore, we propose to quantify disturbances with respect to a cause-effect chain $\mathcal{C}$ in terms of  \textbf{observed WCRTs with added disturbance} ($oWCRT_i'$) 
\begin{align}
	oWCRT_i' := oWCRT_i + \Delta oWCRT_i. 
\end{align}
where we assume $\Delta oWCRT_i > 0$.
Other ways to model disturbances are well possible, but we choose to be consistent with the novel system model developed in Section \ref{sec:model-novel}. \\
\bigskip

In the following, we discuss the problem of how to determine the robustness margins of a cause-effect chain $\mathcal{C} = \left\{ \tau_1, \tau_2, \ldots, \tau_n \right\}$ which is part is part of a given software.
The robustness margins are considered as a help (a) in software design to evaluate its robustness properties, and (b) in software maintenance to predict the potential impact of updates. 
Furthermore, we develop a robustness test to check whether the end-to-end deadline of the cause-effect chain $\mathcal{C}$ is likely to be violated and we base this test on the previously determined robustness margins.


\subsection{Derivation of Robustness Margins and a Robustness Test}
We begin this section by presenting the underlying idea of the robustness margins of a cause-effect chain $\mathcal{C}$ and the related robustness test.
\smallskip

As we have seen in Section \ref{sec:cec-analysis}, the maximum end-to-end latency of a cause-effect chain $\mathcal{C}$ is computed by (1) enumerating all possible data paths between jobs of the cause-effect chain (reachability problem) and (2) finding the data path with the longest end-to-end latency.
The principle of the \emph{robustness test} is to check, whether additional data paths between jobs can potentially be created by the increased observed worst-case response times $oWCRT_i'$ after a modification of the software.
Additional data paths are a necessary condition for an increased maximum end-to-end latency of the cause-effect chain $\mathcal{C}$, where the increased end-to-end latency may potentially violate the specified end-to-end deadline if the software is not sufficiently robust. 
A \emph{robustness margin} quantifies the maximum tolerated disturbance, such that no new data path is created in the presence of this disturbance.
\smallskip

The impact of additional data paths on maximum end-to-end latencies of cause-effect chains results from the algorithm for the computation of end-to-end latencies presented in Section \ref{sec:cec-analysis}. 
The size of a read interval $R(\tau_{i,j})$ grows with an increased worst-case response time $oWCRT_i$ (the interval boundary $R^+(\tau_{i,j})$ is shifted to the right by $\Delta oWCRT_i$).
Likewise the size of a data interval $D(\tau_{i,j})$ grows with an increased worst-case response time $oWCRT_i$ (the interval boundary $D^+(\tau_{i,j})$ is shifted to the right by $\Delta oWCRT_i$). 
The larger interval sizes caused by disturbances may lead to more overlap in data intervals and read intervals, and thus to additional data paths.
\smallskip


\subsubsection{Robustness Margin}
Let us focus on a pair of directly communicating tasks $\tau_i \prec \tau_k$ which are part of the  cause-effect chain $\mathcal{C}$.
Without any disturbance let job $\tau_{k,l_j}$ be the latest job of task $\tau_k$ that reads data from job $\tau_{i,j}$.  
As job $\tau_{k,l_j+1}$ cannot read from $\tau_{i,j}$, we have
\begin{align}
	R^-(\tau_{k,l_j+1}) > D^+(\tau_{i,j}). 
\end{align}


We define therefore the \textbf{robustness margin} $RM(\tau_i, \tau_k)$ for the pair of communicating tasks $\tau_i \prec \tau_k$ within a cause-effect chain $\mathcal{C}$ as
\begin{align}
	RM(\tau_i, \tau_k) = 
	\max \left\{0, 
					\min \limits_j \left\{ R^-(\tau_{k,l_j+1} - D^+(\tau_{i,j})) \right\} 
	\right\}.
\end{align}
Note that the number of jobs $\tau_{i,j}$ to be tested is limited due to an existing hyperperiod of periodic cause-effect chains.
The concept of robustness margin is illustrated in Figure \ref{fig:r-margin}.
\smallskip



\begin{figure}[p]
    \centering
    \begin{subfigure}[b]{0.9\textwidth}
        \includegraphics[width=\textwidth]{fig/robustness1.png}
        \caption{\textbf{No disturbance in the system.} \\Job $\tau_{k,l_j}$ is the latest job of task $\tau_k$ that reads data from job $\tau_{i,j}$.}
        \label{fig:robustness1}
    \end{subfigure}
    
		\vspace*{2cm} 

    \begin{subfigure}[b]{0.9\textwidth}
        \includegraphics[width=\textwidth]{fig/robustness2.png}
        \caption{\textbf{Disturbed system.} \\A new data path may be created, if the distance between $D^+(\tau_{i,j})$ and $R^-(\tau_{k,l_j+1}$ (i.e. the robustness margin) becomes zero.}
        \label{fig:robustness2}
    \end{subfigure}

		\vspace*{1cm} 

    \caption{Illustrating the concept of a robustness margin}\label{fig:r-margin}
\end{figure}


\subsubsection{Robustness Test}
Let us repeat here that the principle of the robustness test is to check, whether additional data paths between jobs can potentially be created by the increased observed worst-case response times $oWCRT_i'$ after a modification of the software. 
Data path analysis relies on the (non-)intersection of read intervals and data intervals, as explained in Section \ref{sec:cec-analysis}. 
Therefore, we focus on the impact of disturbances on the size of read intervals and data intervals.
\bigskip

Let us expand the robustness relation from above (condition under which no new data path is created between tasks  $\tau_i \prec \tau_k$) 
\begin{align}
	\forall j: \quad R^-(\tau_{k,l_j+1}) > D^+(\tau_{i,j})
\end{align}
using equations \ref{eq:read-upper-bound} and \ref{eq:data-upper-bound}.
For the case of a nonzero disturbance $\Delta oWCRT_i \neq 0$, we have
\begin{align}
	\forall j: \quad  \Psi_k + ((l_j+1)-1) \cdot T_k > \Psi_i + j \cdot T_i + (oWCRT_i + \Delta oWCRT_i).
\end{align}
\bigskip

If we rearrange the inequation such that $\Delta oWCRT_i$ is on the left hand side, we have a necessary test condition stating how large the disturbance $\Delta oWCRT_i$ can be without creating an additional data path between the pair of communicating tasks $\tau_i \prec \tau_k$
\begin{align}
	& 
	&&\forall j: \quad \Delta oWCRT_i  < \Psi_k + l \cdot T_k -  \Psi_i - j \cdot T_i  - oWCRT_i \\ 
	& \Leftrightarrow 
	&& \Delta oWCRT_i  < \min \limits_j \left\{ \Psi_k + l \cdot T_k -  \Psi_i - j \cdot T_i  - oWCRT_i \right\} \\
	& \Leftrightarrow 	
	&& \Delta oWCRT_i  < RM(\tau_i, \tau_k).
\end{align}


\subsubsection{Applying the Robustness Test} The robustness test with its necessary test condition can be used as check whether a software modification is bound to increase the maximum end-to-end latency of a cause-effect chain. 
The steps below summarize how to perform the robustness test for a cause-effect chain $\mathcal{C}$.
\begin{enumerate}
	\item Simulate or measure $oWCRT_i$ for all $\tau_i \in \mathcal{C}$ with respect to the original software.
	\item Simulate or measure $oWCRT_i'$ for all $\tau_i \in \mathcal{C}$ with respect to the modified software.
	\item Derive 
	$\Delta oWCRT_i = oWCRT_i' - oWCRT_i$ for all $\tau_i \in \mathcal{C}$.
	\item For each directly communicating pair of tasks in the cause-effect chain $\mathcal{C} = \left\{ \tau_1, \tau_2, \ldots, \tau_n \right\}$, that is $\tau_1, \tau_2$, $\tau_2, \tau_3$ etc.,  check the test condition
\begin{align}
	\Delta oWCRT_i  < RM(\tau_i, \tau_k).
\end{align}	
	\item If any of the tests is evaluated to False, the algorithm to compute maximum end-to-end latencies as stated in Section \ref{sec:cec-analysis} has be re-run for deadline verification.
\end{enumerate}


